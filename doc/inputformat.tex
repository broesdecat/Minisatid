\documentclass{article}

\usepackage{amsmath}
\usepackage{graphicx}
\usepackage{amssymb}
\usepackage{color}
\usepackage{xspace}
\usepackage{url}

\title{On the input language of MinisatID}
\author{Broes De Cat}

\begin{document}

\maketitle

\section{Supported formats}
Several input formats are supported:
\begin{description}
	\item[opb] open pseudo-boolean competition format, defined in www.cril.univ-artois.fr/PB11/format.pdf
	\item[cnf] dimacs clausal normal form, defined in www.satcompetition.org/2011/rules.pdf
	\item[asp] ground answer set program format, defined in www.tcs.hut.fi/Software/smodels/lparse.ps.gz
	\item[qbf] quantified boolean formula format, defined in www.qbflib.org/Draft/qDimacs.ps.gz
	\item[ecnf] see the BNF grammar below
	\item[mecnf] experimental extension for modal operators (allows to express that, given a partial interpretation, a subtheory has a model / all extensions are models).
\end{description}

From an implementation point of view, the formats cnf, qbf, ecnf and mecnf are detected automatically. The formats opb and asp have to be specified by the option "{\verb --format=x }", with x="{\verb opb }", respectively x="{\verb asp }".

\section{ECNF format}
ECNF or Extended Clausal Normal Form is the native input language of the solver MinisatID.

\subsection{Syntax}
The syntax of the language can be expressed as the following (non-formal) BNF grammar ("+" expresses one or more, "*" expresses zero or more):

\begin{verbatim}
<formula>   ::= <comment>* "p ecnf" <header>* <EOL> <ecnftheory>*
<comment>   ::= "c" <char (not EOL)>* <EOL>
<header>    ::= "def" | "aggr" | "mnmz"

<ecnftheory>::= <comment>|<clause>|<rule>|<agg>|<set>|<wset>|
                <mnmz>|<subsetmnmz>|<aggmnmz>

<clause>    ::= <lit>* <end>

<sem>       ::= "D"|"C"
<rule>      ::= <sem> <atom> <lit>* <end>

<set>       ::= "Set " <id> <lit>+ <end>
<wset>      ::= "WSet " <id> <tuple>+ <end>

<operator>  ::= "L"|"G"
<aggtype>   ::= "Sum" | "Min" | "Max" | "Prod" | "Card"
<agg>       ::= <aggtype> <sem> <operator> <atom> <id> <int> <end>

<mnmz>      ::= "Mnmlist " <lit>+ <end>
<subsetmnmz>::= "Mnmsubset " <lit>+ <end>
<aggmnmz>   ::= "Mnmagg " <aggtype> <atom> <id> <end>

<tuple>     ::= <lit>"="<int>		
<atom>      ::= <posint>
<lit>       ::= "-"<posint>|<posint>
<id>        ::= <posint>
<posint>    ::= any non-zero positive integer
<int>       ::= any integer

<end>       ::= "0"
\end{verbatim}

Additional constraints:
\begin{itemize}
	\item Each element is separated by at least one whitespace. Extra whitespaces or newlines can be added anywhere except in the middle of comments.
	\item Any atom can only be the head of one rule or aggregate.
	\item Any id can only be defined once.
	\item A set has to come before any aggregate using it.
\end{itemize}
	
\subsection{Semantics}

\paragraph{Literal} { \verb <lit> } a literal is an atom (positive integer) or its negation (negative integer)

\paragraph{Clause} 
\begin{verbatim}
<clause>    ::= <lit>* <end>
\end{verbatim} 
a clause is the disjunction of its literals 

\paragraph{Rule}
\begin{verbatim}
<sem>       ::= "D"|"C"
<rule>      ::= <sem> <atom> <lit>* <end>
\end{verbatim} 
A rule defines its atom in terms of a disjunction (sem="D") or a conjunction (sem="C") of a set of literals. All rules belong to the same definition and are evaluated according to the well-founded semantics (stable semantics can also be specified).

\paragraph{Sets}
\begin{verbatim}
<set>       ::= "Set " <id> <lit>+ <end>
<wset>      ::= "WSet " <id> <tuple>+ <end>
\end{verbatim} 
A set is identified by an id and is a non-empty set of tuples literal=integer. The first notation is an abbreviation of literal=1. It signifies that the literal has the associated weight if it is true, and does not belong to the set if it is false.

\paragraph{Aggregates}
\begin{verbatim}
<sem>       ::= "D"|"C"
<operator>  ::= "L"|"G"
<aggtype>   ::= "Sum" | "Min" | "Max" | "Prod" | "Card"
<agg>       ::= <aggtype> <sem> <operator> <atom> <id> <int> <end>
\end{verbatim} 
Given an example expression
\begin{verbatim}
WSet 1 1=2 -2=3 0
Sum D G 4 1 3 0
\end{verbatim}
The semantics is that the atom 4 is defined by the aggregate expression $sum(set_1) \geq 3$. 3 is called the \emph{bound}, 4 is the \emph{head}. If sem="C", completion semantics is used, if sem="D", well-founded semantics (with recursive aggregates) is used.
The available aggregate types are sum (also called pseudo-boolean / linear constraint), cardinality (count), maximum, minimum and product. The operator is either "G", $aggregate \geq bound$, or "L", $aggregate \leq bound$.
 
\paragraph{Optimization}
Optimization can be one of three types:
\begin{itemize}
\item Minimize ordered list ({ \verb <mnmz> }): given a list of literals, ordered from best (left) to worst, a model is better if it makes a better literal true.  
\item Subsetminimization ({ \verb <subsetmnmz> }): find the smallest subset of the set of literals for which a model of the theory exists.
\item Aggregate optimization ({ \verb <aggmnmz> }): minimize the value of the aggregate, given its type, its head and its set.
\end{itemize}

\section{License}
Copyright 2007-2011 Katholieke Universiteit Leuven

Use of this software is governed by the GNU LGPLv3.0 license

Written by Broes De Cat and Maarten Mariën, K.U.Leuven, Departement Computerwetenschappen, Celestijnenlaan 200A, B-3001 Leuven, Belgium
\end{document}
